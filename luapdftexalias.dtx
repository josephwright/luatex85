% \iffalse
%% Source File: luapdftexalias.dtx
%% Copyright 2015 LaTeX3 project
%%
%% This file may be distributed under the terms of the LPPL.
%% See README for details.
%
%<*dtx>
          \ProvidesFile{luapdftexalias.dtx}
%</dtx>
%<package>\ifx\ProvidesPackage\undefined
%<package>\def\next#1#2[#3]{\wlog{#2 #3}}
%<package>\expandafter\next\fi
%<package>\ProvidesPackage{luapdftexalias}
%<driver> \ProvidesFile{luapdftexalias.drv}
% \fi
%         \ProvidesFile{luapdftexalias.dtx}
       [2015/12/05 v0.01 pdftex aliases for luatex]
%
% \iffalse
%<*driver>
\documentclass{ltxdoc}
\begin{document}
\DocInput{luapdftexalias.dtx}
\end{document}
%</driver>
% \fi
%
% \hfuzz0.6pt
%
% \GetFileInfo{luapdftexalias.dtx}
%
% \title{The \textsf{luapdftexalias} Package\thanks{This file
%        has version number \fileversion, last
%        revised \filedate.
% Please report any issues at https://github.com/josephwright/luatex85/issues}}
% \author{LaTeX3 project}
% \maketitle
%
% \section{Introduction}
%
% Lua\TeX\ 0.85 and 0.87 contain many changes from Lua\TeX~0.80 as contained in \TeX{}Live 2014.
% Most notably almost all the pdf\TeX\ extended primitves with names
% \verb|\pdf...| have been renamed or removed. Lua\TeX\ is aiming for
% a cleaner separation of the ``back end'' PDF generation (that
% corresponds to the work of a dvi driver with classical \TeX.
% 
% There are many other changes and bug fixes in the Lua\TeX\ sources,
% however this package is just concerned with compatibility for
% documents or packages using the pdf\TeX\ primitives.
%
% The changes are of several types:
%
% A few commands have been removed,
% as the facilities are achievable in lua (mostly these had already
% been removed in earlier release).
%
% Some commands have been ``adopted'' as Lua\TeX\ primitives and so
% lose their `verb|\pdf|` prefix (and in some cases are renamed)
% so \verb|pdfsavepos| becomes \verb|savepos|, but \verb|\pdfoutput|
% becomes \verb|outputmode|.
%
% The majority of the ``back end'' commands have been removed and
% replaced by calls to one of three new primitives,
% \verb|\pdffeedback|, \verb|\pdfextension| and \verb|\pdfvariable|
% These take keywords so for example \verb|\pdfliteral| becomes
% \verb|\pdfextension literal|.
%
% The Lua\TeX\ manual lists suitable compatibility definitions
% that may be made so that documents can continue to use the old
% names. Mostly this package just consists of those definitions, with
% minor changes in some cases. (Mostly different choices over the use
% of \verb|\protected| or \verb|\edef|.)
%
% In general it is recommended that any package is updated to use the
% new primitive Lua\TeX\ syntax when used with Lua\TeX, but until
% packages are updated authors may find that adding\\%
% \verb|\RequirePackage{luapdftexalias}|\\
% as the first line of their
% document helps with the use of older packages with the new LuaTeX.
%
% As noted above there are other changes in Lua\TeX, notably the
% removal of the |\verb\write18| syntax for accessing system commands,
% a small companion package is distributed with this package. Rather
% than load this package directly you may prefer\\
% \verb|\RequirePackage{luatex85}|\\
% This loads this package, the textsf{shellesc} package which emulates
% \verb|\write18| and includes any other smaller compatibility
% workarounds as required.
%
% Note that if packages are found that require \textsf{luatex85}
% you may want to contact the authors asking that the packages 
% be updated to current Lua\TeX\ syntax. The \textsf{luatex85}
% should be seen as a temporary aid to improve compatibility during
% the transition towards Lua\TeX~1.0 it is not intended that future
% documents should always have to load this compatibility emulation.
%
% The package is designed to also be usable with plain Lua\TeX.
%
% 
% \section{Implementation}
%
%    \begin{macrocode}
%<*package>
%    \end{macrocode}
%
% \subsection{Checking the engine}
%    \begin{macrocode}
\ifx\pdfvariable\undefined
  \expandafter\endinput
\fi
%    \end{macrocode}

% \subsection{Commands promoted to Lua\TeX\ primitives.}
%
%
%    \begin{macrocode}
\let\pdfpagewidth        \pagewidth
\let\pdfpageheight       \pageheight
\let\pdfadjustspacing    \adjustspacing
\let\pdfprotrudechars    \protrudechars
\let\pdfnoligatures      \ignoreligaturesinfont
\let\pdffontexpand       \expandglyphsinfont
\let\pdfcopyfont         \copyfont
\let\pdfxform            \saveboxresource
\let\pdflastxform        \lastsavedboxresourceindex
\let\pdfrefxform         \useboxresource
\let\pdfximage           \saveimageresource
\let\pdflastximage       \lastsavedimageresourceindex
\let\pdflastximagepages  \lastsavedimageresourcepages
\let\pdfrefximage        \useimageresource
\let\pdfsavepos          \savepos
\let\pdflastxpos         \lastxpos
\let\pdflastypos         \lastypos
\let\pdfoutput           \outputmode
\let\pdfdraftmode        \draftmode
\let\pdfpxdimen          \pxdimen
\let\pdfinsertht         \insertht
\let\pdfnormaldeviate    \normaldeviate
\let\pdfuniformdeviate   \uniformdeviate
\let\pdfsetrandomseed    \setrandomseed
\let\pdfrandomseed       \randomseed
\let\pdfprimitive        \primitive
\let\ifpdfprimitive      \ifprimitive
\let\ifpdfabsnum         \ifabsnum
\let\ifpdfabsdim         \ifabsdim
%    \end{macrocode}
%
% \subsection{Commands converted to \cs{pdffeedback}}
% Expandable commands use a sipmple |\def| internal registers that
% were accessed via |\the| in PDF\TeX\ use a |\protected| definition 
% using |\numexpr| terminated by an explicit |\relax|.
%    \begin{macrocode}
\protected\def\pdftexversion     {\numexpr\pdffeedback version\relax}
          \def\pdftexrevision    {\pdffeedback revision}
\protected\def\pdflastlink       {\numexpr\pdffeedback lastlink\relax}
\protected\def\pdfretval         {\numexpr\pdffeedback retval\relax}
\protected\def\pdflastobj        {\numexpr\pdffeedback lastobj\relax}
\protected\def\pdflastannot      {\numexpr\pdffeedback lastannot\relax}
          \def\pdfxformname      {\pdffeedback xformname}
          \def\pdfcreationdate   {\pdffeedback creationdate}
          \def\pdffontname       {\pdffeedback fontname}
          \def\pdffontobjnum     {\pdffeedback fontobjnum}
          \def\pdffontsize       {\pdffeedback fontsize}
          \def\pdfpageref        {\pdffeedback pageref}
          \def\pdfcolorstackinit {\pdffeedback colorstackinit}
%    \end{macrocode}
%
% \subsection{Commands converted to calls to \cs{pdfextension}}
% These use a |\protected| definition. Comands that take no following
% argument are currently terminated by |\releax| as suggested in the
% Lua\TeX\ manual, although it would be appear to be sufficient to
% consistently terminate these commands with a space.
%    \begin{macrocode}
\protected\def\pdfliteral        {\pdfextension literal}
\protected\def\pdfcolorstack     {\pdfextension colorstack}
\protected\def\pdfsetmatrix      {\pdfextension setmatrix}
\protected\def\pdfsave           {\pdfextension save\relax}
\protected\def\pdfrestore        {\pdfextension restore\relax}
\protected\def\pdfobj            {\pdfextension obj }
\protected\def\pdfrefobj         {\pdfextension refobj }
\protected\def\pdfannot          {\pdfextension annot }
\protected\def\pdfstartlink      {\pdfextension startlink }
\protected\def\pdfendlink        {\pdfextension endlink\relax}
\protected\def\pdfoutline        {\pdfextension outline }
\protected\def\pdfdest           {\pdfextension dest }
\protected\def\pdfthread         {\pdfextension thread }
\protected\def\pdfstartthread    {\pdfextension startthread }
\protected\def\pdfendthread      {\pdfextension endthread\relax}
\protected\def\pdfinfo           {\pdfextension info }
\protected\def\pdfcatalog        {\pdfextension catalog }
\protected\def\pdfnames          {\pdfextension names }
\protected\def\pdfincludechars   {\pdfextension includechars }
\protected\def\pdffontattr       {\pdfextension fontattr }
\protected\def\pdfmapfile        {\pdfextension mapfile }
\protected\def\pdfmapline        {\pdfextension mapline }
\protected\def\pdftrailer        {\pdfextension trailer }
\protected\def\pdfglyphtounicode {\pdfextension glyphtounicode }
%    \end{macrocode}
%
% \subsection{Commands converted to calls to \cs{pdfvariable}}
% Currently as suggested in the manual the call to |\pdfvariable| has
% no explict termination, and relies on the fact that no variable name
% is a prefix of another. |\edef| is used to save one expansion step
% when these comands are used the definition directly access the
% internal command token.
%    \begin{macrocode}
\protected\edef\pdfcompresslevel       {\pdfvariable compresslevel}
\protected\edef\pdfobjcompresslevel    {\pdfvariable objcompresslevel}
\protected\edef\pdfdecimaldigits       {\pdfvariable decimaldigits}
\protected\edef\pdfgamma               {\pdfvariable gamma}
\protected\edef\pdfimageresolution     {\pdfvariable imageresolution}
\protected\edef\pdfimageapplygamma     {\pdfvariable imageapplygamma}
\protected\edef\pdfimagegamma          {\pdfvariable imagegamma}
\protected\edef\pdfimagehicolor        {\pdfvariable imagehicolor}
%    \end{macrocode}
% Note that |\pdfimageaddfilename| was never actually in PDF\TeX,
% But is included here so that all the |\pdfvariable| cases are covered.
%    \begin{macrocode}
\protected\edef\pdfimageaddfilename    {\pdfvariable imageaddfilename}
%    \end{macrocode}
%    \begin{macrocode}
\protected\edef\pdfpkresolution        {\pdfvariable pkresolution}
\protected\edef\pdfinclusioncopyfonts  {\pdfvariable inclusioncopyfonts}
\protected\edef\pdfinclusionerrorlevel {\pdfvariable inclusionerrorlevel}
%    \end{macrocode}
% Note that |\pdfreplacefont| was never actually in public releases of
% PDF\TeX, It was in Lua\TeX0.85, but discussion on luatex list
% lead to it being removed in 0.87
%    \begin{macrocode}
%\protected\edef\pdfreplacefont         {\pdfvariable replacefont}
%    \end{macrocode}
%    \begin{macrocode}
\protected\edef\pdfgentounicode        {\pdfvariable gentounicode}
\protected\edef\pdfpagebox             {\pdfvariable pagebox}
\protected\edef\pdfminorversion        {\pdfvariable minorversion}
\protected\edef\pdfuniqueresname       {\pdfvariable uniqueresname}
\protected\edef\pdfhorigin             {\pdfvariable horigin}
\protected\edef\pdfvorigin             {\pdfvariable vorigin}
\protected\edef\pdflinkmargin          {\pdfvariable linkmargin}
\protected\edef\pdfdestmargin          {\pdfvariable destmargin}
\protected\edef\pdfthreadmargin        {\pdfvariable threadmargin}
\protected\edef\pdfpagesattr           {\pdfvariable pagesattr}
\protected\edef\pdfpageattr            {\pdfvariable pageattr}
\protected\edef\pdfpageresources       {\pdfvariable pageresources}
%    \end{macrocode}
% Note that |\pdfxformattr| and |\pdfxformresources| were never
% actually in PDF\TeX, But are included here so that all the
% |\pdfvariable| cases are covered.
%    \begin{macrocode}
\protected\edef\pdfxformattr           {\pdfvariable xformattr}
\protected\edef\pdfxformresources      {\pdfvariable xformresources}
%    \end{macrocode}
%    \begin{macrocode}
\protected\edef\pdfpkmode              {\pdfvariable pkmode}
%    \end{macrocode}
%
%    \begin{macrocode}
%</package>
%    \end{macrocode}
%
% \Finale
%
