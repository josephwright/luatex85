\RequirePackage{luatex85}
% not actually bad here
\begin{filecontents}{\jobname.spell.bad}
typesetting
manuscript
computer
\end{filecontents}

\ifx\luaescapestring\undefined
\let\luaescapestring\luatexluaescapestring
\fi

\documentclass{article}

% four packages using luatexbase/luaotfload
\usepackage{polyglossia}
\setmainlanguage{english}\setotherlanguages{french,tibetan}
\newfontfamily\tibetanfont{Microsoft Himalaya}% or any other suitable font
\usepackage{spelling}
\usepackage{chickenize}

% oops:-)
\def\unuppercasecolor{
  \directlua{luatexbase.remove_from_callback("post_linebreak_filter","uppercasecolor")}}


\usepackage{fontspec}
\setmainfont{TeX Gyre Pagella}


\begin{document}

% `Start of the TeXBook
\boustrophedon

\noindent
{\sc Gentle} {\sc Reader}: \strut This is a handbook about
\TeX, a new typesetting system intended for the creation
of beautiful books---and especially for books that contain a lot of
mathematics. By preparing a manuscript in \TeX\ format, you will be
telling a computer exactly how the manuscript is to be transformed into
pages whose typographic quality is comparable to that of the world's
finest printers; yet you won't need to do much more work than would be
involved if you were simply typing the manuscript on an ordinary
typewriter. In fact, your total work will probably be significantly less,
if you consider the time it ordinarily takes to revise a typewritten manuscript,
since computer text files are so easy to change and to reprocess. \
(If such claims sound too good to be true, keep in mind that they were made
by \TeX's designer, on a day when \TeX\ happened to
be working, so the statements may be biased; but read on anyway.)

\unboustrophedon

\chickenize
one two three four five six seven eight nine ten

\unchickenize

\randomcolor
Some random text in random colours.

\unrandomcolor

\uppercasecolor
Some Text with Mixed Case LaTeX Input.

\unuppercasecolor


\paragraph{French} should get extra space:

foo: bar; foobar!

\begin{french}foo: bar; foobar!\end{french}

\paragraph{Tibetan}should break at tshegs

\begin{tibetan}\raggedright
འངངས་འངངས་འངངས་འངངས་འངངས་འངངས་འངངས་འངངས་འངངས་འངངས་འངངས་འངངས་འངངས་འངངས་འངངས་འངངས་འངངས་འངངས་འངངས་འངངས་འངངས་འངངས་འངངས་འངངས་འངངས་འངངས་འངངས་འངངས་འངངས་འངངས་འངངས་འངངས་འངངས་འངངས་འངངས་འངངས་འངངས་འངངས་འངངས་འངངས་འངངས་འངངས་འངངས་འངངས་འངངས་འངངས་འངངས་འངངས་འངངས་འངངས་འངངས་འངངས་འངངས་འངངས་འངངས་འངངས་འངངས་འངངས་འངངས་འངངས་འངངས་འངངས་འངངས་འངངས་འངངས་འངངས་འངངས་འངངས་འངངས་འངངས་འངངས་འངངས་འངངས་འངངས་འངངས་འངངས་འངངས་འངངས་འངངས་འངངས་འངངས་འངངས་འངངས་འངངས་འངངས་འངངས་འངངས་འངངས་འངངས་འངངས་འངངས་འངངས་འངངས་འངངས་འངངས་འངངས་འངངས་འངངས་འངངས་འངངས་འངངས་འངངས་འངངས་འངངས་འངངས་འངངས་འངངས་འངངས་འངངས་འངངས་འངངས་

\end{tibetan}
\end{document}
